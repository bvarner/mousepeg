%HHHHHHHHHHHHHHHHHHHHHHHHHHHHHHHHHHHHHHHHHHHHHHHHHHHHHHHHHHHHHHHHHHHHHHHHHH

\section{Getting more realistic}

%HHHHHHHHHHHHHHHHHHHHHHHHHHHHHHHHHHHHHHHHHHHHHHHHHHHHHHHHHHHHHHHHHHHHHHHHHH

Our example so far is not very realistic.
Any decent parser will allow white space in the input.
This is simply achieved in PEG by defining \Space\
as a sequence of zero or more blanks,
and inserting \Space\ in strategic places.
We are going to do this, and on this occasion add two features:
an optional minus sign in front of the first \Number,
and an alternative minus between the \Number s:

\small
\begin{Verbatim}[frame=single,framesep=2mm,samepage=true,xleftmargin=15mm,xrightmargin=15mm,baselinestretch=0.8]
   Sum     = Space Sign Number (AddOp Number)* !_ {sum} ;
   Number  = Digits Space {number} ; 
   Sign    = ("-" Space)? ;
   AddOp   = [-+] Space ;
   Digits  = [0-9]+ ;
   Space   = " "* ;
\end{Verbatim}
\normalsize

The semantic action \Numbera\ will now see this configuration
of the "right-hand side":

\small
\begin{Verbatim}[samepage=true,xleftmargin=15mm,baselinestretch=0.9]
 Digits Space
   0      1
\end{Verbatim}
\normalsize

The text to be converted is now represented by \tx{rhs(0)} rather than \tx{lhs()},
so \Numbera\ appears as follows:

\small
\begin{Verbatim}[frame=single,framesep=2mm,samepage=true,xleftmargin=15mm,xrightmargin=15mm,baselinestretch=0.8]
   //-------------------------------------------------------------------
   //  Number = Digits Space
   //             0      1
   //-------------------------------------------------------------------
   void number()
     { lhs().put(Integer.valueOf(rhs(0).text())); }
\end{Verbatim}
\normalsize

Note that we could have defined \Number\ as \tx{[0-9]+~Space},
but this would present \Numbera\ with a "right-hand side" of variable length,
and give it a job of collecting all the digits.
It is much easier to let the parser do the job.

\newpage
The semantic action \Suma\ will see this configuration
of the "right-hand side":

\small
\begin{Verbatim}[samepage=true,xleftmargin=15mm,baselinestretch=0.9]
 Space Sign Number AddOp Number ... AddOp Number
   0     1    2      3     4         n-2   n-1
\end{Verbatim}
\normalsize

where $n = 2p + 3$ for $p = 0, 1, 2, \ldots\;$.

The text for \Sign\ is either an empty string
or a minus followed by zero or more blanks.
You can check which is the case with 
the helper method \tx{isEmpty()}.
The text for \tx{AddOp} is either a plus or a minus  
followed by zero or more blanks.
You can obtain the first character of this text with
the helper method \tx{charAt()}.
This gives the following new version of \Suma:

\smallskip
\small
\begin{Verbatim}[frame=single,framesep=2mm,samepage=true,xleftmargin=15mm,xrightmargin=15mm,baselinestretch=0.8]
   //-------------------------------------------------------------------
   //  Sum = Space Sign Number AddOp Number ... AddOp Number
   //          0     1    2      3     4         n-2   n-1
   //-------------------------------------------------------------------
   void sum()
     {
       int n = rhsSize();
       int s = (Integer)rhs(2).get();
       if (!rhs(1).isEmpty()) s = -s;
       for (int i=4;i<n;i+=2)
       {
         if (rhs(i-1).charAt(0)=='+')
           s += (Integer)rhs(i).get();
         else
           s -= (Integer)rhs(i).get();
       }
       System.out.println(s);
     }
\end{Verbatim}
\normalsize

Note again that we could have \tx{("-"~Space)?} instead of \tx{Sign} in the definition of \Sum\,.
But, according to the rule given in Section~\ref{RHS}, 
\Suma\ would then have to handle two different configuration
of the "right-hand side": 
one starting with (\tx{Space~"-"~Space~Number}...)
and one starting with (\tx{Space~Number}...).
In our definition of \Sum, \Sign\ is always there, even if the underlying (\tx{"-"~Space}) failed.
Note that \Space\ is also present even if it did not consume any blanks.

The above illustrates the fact that you have to exercise a lot of care when defining 
rules expected to have semantic actions.
Introducing \Digits\ and \Sign\ as separate named expressions
made the "right-hand side" in each case easier to handle.
For reasons that should now be obvious, you are not recommended
to have repetitions or optional expressions inside repetitions.
A clever programming of semantic action can always handle them,
but the job is much better done by the parser.

You find the grammar and the semantics class for this example
as \tx{myGrammar.txt} and \tx{mySemantics.java} in \tx{example3}.
Copy them to the \tx{work} directory (replacing the old ones),
generate the parser, and compile it as before.
A session with the new parser may look like this:

\small
\begin{Verbatim}[samepage=true,xleftmargin=15mm,baselinestretch=0.8]
java mouse.TryParser -P myParser
>   17 + 4711
4728
> -1 + 2 - 3
-2
>
\end{Verbatim}
\normalsize

To close this section, note that you implemented here some functions
traditionally delegated to a separate "lexer": conversion of integers
and ignoring white space.
If you wanted, you could easily include newlines and comments in \Space.




