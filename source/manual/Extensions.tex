\newpage
%HHHHHHHHHHHHHHHHHHHHHHHHHHHHHHHHHHHHHHHHHHHHHHHHHHHHHHHHHHHHHHHHHHHHHHHHHH

\section{Extensions to PEG\label{Extensions}}

%HHHHHHHHHHHHHHHHHHHHHHHHHHHHHHHHHHHHHHHHHHHHHHHHHHHHHHHHHHHHHHHHHHHHHHHHHH

\Mouse\ accepts the following four forms of expression 
that are not part of the standard PEG:

\medskip
\begin{tabular}{|c|p{0.85\linewidth}|}
\hline
\ \ $e_1\tx{*+}\,e_2$\ \ \upsp
   & Shorthand for \tx{(!}$e_2\,e_1$\tx{)*}\,$e_2$:
     iterate $e_1$ until $e_2$.\dnsp\\
\hline
$e_1\tx{++}\,e_2$\upsp\newline
   & Shorthand for \tx{(!}$e_2\,e_1$\tx{)+}\,$e_2$:
     iterate $e_1$ at least once until $e_2$.\dnsp\\
\hline
\ \ $e_1\tx{:}\,e_2$\ \ \upsp
   & Invoke $e_1$.
     If it succeeds, invoke $e_2$ on the string consumed by~$e_1$.
     Indicate succes if it consumes exactly that string.
     Otherwise reset input as it was before the invocation of $e_1$
     (do not consume anything) and indicate failure.\dnsp\\
\hline
\ \ $e_1\tx{:!}\,e_2$\ \ \upsp
   & Invoke $e_1$.
     If it succeeds, invoke $e_2$ on the string consumed by~$e_1$.
     Indicate succes if does not consume exactly that string.
     Otherwise reset input as it was before the invocation of $e_1$
     (do not consume anything) and indicate failure.\dnsp\\
\hline
\end{tabular}

\bigskip
Using "\tx{*+}" and "\tx{++}", the definitions

\small
\begin{Verbatim}[samepage=true,xleftmargin=15mm,baselinestretch=0.8]
Comment = "/*" (!"*/" _)* "*/" ;
Input   = (!EOF Line)+ EOF ;
String  = ["] (!["] Char)+ ["] ;
\end{Verbatim}
\normalsize

are rewritten as 

\small
\begin{Verbatim}[samepage=true,xleftmargin=15mm,baselinestretch=0.8]
Comment = "/*" _*+ "*/" ;
Input   = Line++ EOF ;
String  = ["] Char++ ["] ;
\end{Verbatim}
\normalsize

which is not only easier to read, but also has a better implementation.

\medskip
The operators "\tx{:}" and "\tx{:!}" are intended to simplify definition of reserved words.

Suppose you want to define \tx{Identifier} as any sequence 
of letters other than \tx{"int"} or \tx{"float"}.
Using pure PEG, you can do it as follows, 
with \tx{INT} and \tx{FLOAT} used to denote these keywords in your syntax:

\small
\begin{Verbatim}[samepage=true,xleftmargin=15mm,xrightmargin=15mm,baselinestretch=0.8]
Identifier = !Keyword Letter+ ;
Keyword = ("int"/"float") !Letter ;
INT   = "int" !Letter ;
FLOAT = "float" !Letter ;
\end{Verbatim}
\normalsize

The predicate \tx{!Letter} is needed because
\tx{!("int"/"float")} fails on strings like \tx{internal} or \tx{floater},
which are thus not recognized as identifiers,
while their prefixes are falsely recognized as keywords.

Using the "colon" operators you define the same like this:

\small
\begin{Verbatim}[samepage=true,xleftmargin=15mm,xrightmargin=15mm,baselinestretch=0.8]
Word = Letter+ ;
Identifier = Word:!("int"/"float") ;
INT   = Word:"int" ;
FLOAT = Word:"float" ;
\end{Verbatim}
\normalsize

It is more clear and, more important, facilitates the inspection by \textsl{PEG Explorer}.

\medskip
All four operators have precedence 4 (cf. table in Section \ref{PEG}).

To obtain the "right-hand side" described in Section \ref{RHS},
replace \tx{*+} and \tx{++} by, respectively, \tx{*} and \tx{+} 
in the construction outlined there.

The second expression of  $e_1$\tx{:}$e_2$ and $e_1\tx{:!}e_2$ does not appear in the "right-hand side".
The expression together with the preceding operator should be removed in the desctibed construction.

Note: reserved words can also be defined with the help of Boolean actions, described under \ref{BoolAct}, 
but their definition is not visible in the syntax and is beyond the reach of \textsl{PEG Explorer}.
This is, however, the only way if the keywords depend on the preceding context like in,
for example, languages C and C++.

