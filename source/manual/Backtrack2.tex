%HHHHHHHHHHHHHHHHHHHHHHHHHHHHHHHHHHHHHHHHHHHHHHHHHHHHHHHHHHHHHHHHHHHHHHHHHH

\section{Backtracking again\label{back2}}

%HHHHHHHHHHHHHHHHHHHHHHHHHHHHHHHHHHHHHHHHHHHHHHHHHHHHHHHHHHHHHHHHHHHHHHHHHH

The grammar just constructed has another non-$LL(1)$ choice
in addition to the choice between two kinds of number.
It is the choice between \tx{Store} and \tx{Print} in \Input: 
they can both start with \tx{Name}.

If you enter something like "\tx{lambda * 7}",
the parser tries \tx{Store} first and processes "\tx{lambda }",
expecting to find equal sign next.
Finding "\tx{*}" instead, the parser backtracks and tries \tx{Print}.
It eventually comes to process "\tx{lambda }" via \tx{Sum}, \tx{Product},
and \tx{Factor}.
To watch this activity, you may generate test version of the parser using option \tx{-T},
as described in Section~\ref{back} and try it.
A possible result is shown below.

\small
\begin{Verbatim}[samepage=true,xleftmargin=15mm,xrightmargin=15mm,baselinestretch=0.75]
  java mouse.TestParser -PmyParser -d

  > lambda = 12

  60 calls: 29 ok, 30 failed, 1 backtracked.
  8 rescanned.
  backtrack length: max 2, average 2.0.

  Backtracking, rescan, reuse:

  procedure        ok  fail  back  resc reuse totbk maxbk at
  ------------- ----- ----- ----- ----- ----- ----- ----- --
  Factor_0          0     1     1     0     0     2     2 After 'lambda = '
  Digits            2     2     0     2     0     0     0
  "."               0     2     0     1     0     0     0
  [0-9]             4     4     0     5     0     0     0

  > lambda * 7
  84.0

  86 calls: 40 ok, 44 failed, 2 backtracked.
  20 rescanned.
  backtrack length: max 7, average 4.0.

  Backtracking, rescan, reuse:

  procedure        ok  fail  back  resc reuse totbk maxbk at
  ------------- ----- ----- ----- ----- ----- ----- ----- --
  Store             0     0     1     0     0     7     7 At start
  Factor_0          0     2     1     0     0     1     1 After 'lambda * '
  Digits            2     4     0     3     0     0     0
  Name              2     1     0     1     0     0     0
  Space             5     0     0     1     0     0     0
  "."               0     3     0     1     0     0     0
  [0-9]             2     6     0     5     0     0     0
  [a-z]            12     3     0     7     0     0     0
  " "               3     5     0     2     0     0     0

  >
\end{Verbatim}
\normalsize

This is quite a lot of rescanning; you may try to see the effects of specifying
"\tx{-m1}":
the number of rescans is reduced to 2 for each input.
