%HHHHHHHHHHHHHHHHHHHHHHHHHHHHHHHHHHHHHHHHHHHHHHHHHHHHHHHHHHHHHHHHHHHHHHHHHH

\section{Checking the grammar\label{Checking}}

%HHHHHHHHHHHHHHHHHHHHHHHHHHHHHHHHHHHHHHHHHHHHHHHHHHHHHHHHHHHHHHHHHHHHHHHHHH

Before generating the parser, \Mouse\ performs a number of checks on the grammar,
and issues messages about its findings.
Using the \Mouse\ tool \tx{TestPEG},
you can perform the same checking and obtain the same diagnostics
without generating the parser.
By specifying option \tx{-D} to the tool, you can also display the grammar.

As an example, below is the result of
\tx{TestPEG} for the grammar:
\small
\begin{Verbatim}[frame=single,framesep=2mm,samepage=true,xleftmargin=15mm,xrightmargin=15mm,baselinestretch=0.8]
  A = (!"a")* !_ / A "b"? ;
  B = B "a" ;
\end{Verbatim}
\normalsize

\medskip
\small
\begin{Verbatim}[samepage=true,xleftmargin=15mm,baselinestretch=0.8]
C:\Users\Giraf\Work>java mouse.TestPEG -G bad.peg -DL
Error: B is void.
Error: B "a" is void.
Error: the grammar has cycle inolving A.
Info: recursion class of B is not used.
Error: argument of (!"a")* is nullable.

  2 rules
  6 unnamed
  4 terminals
  4 left-recursive expressions in 2 classes

Rules
  A = (!"a")* !_ / A "b"? ;   //  1 rec class A
  B = B "a" ;   //  v rec class B

Inner
  (!"a")* !_   //  1ef
  (!"a")*   //  0
  !"a"   //  0f
  A "b"?   //  1 rec class A
  "b"?   //  01
  B "a"   //  v rec class B

Terminals
  "a"   //  1f
  !_   //  1ef
  "b"   //  1f
  "a"   //  1f
\end{Verbatim}
\normalsize

An expression being \emph{void} means that it does not generate any string, in the EBNF sense.
\newline
Thus,  \tx{B = B "a"} is void because it can only produce an infinite chain of replacements.
\newline
The notion  of \emph{cycle} was explained in Section~\ref{LeftRec}.
\newline
The symbols appearing as comments on the right have these meanings:
\ul
\item[] \tx{0} = may consume null string;
\item[] \tx{1} = may consume non-null string;
\item[] \tx{f} = may fail;
\item[] \tx{e} = ends processing;
\item[] \tx{v} = void.
\eul


Comment "\tx{rec class A}"  means that expression belongs to the \emph{recursion class} containing \tx{A}.
Recursion classes are sets of left-recursive expressions 
that interact only among themselves.
\tx{TestPEG} with option \tx{-L} displays information about recursion classes:

\small
\begin{Verbatim}[samepage=true,xleftmargin=15mm,baselinestretch=0.8]
java mouse.TestPEG -G bad.peg -L
Recursion class A
  members:
    A (entry) (exit)
    A "b"?
  seeds:
    (!"a")* !_ (in A)

Recursion class B
  members:
    B
    B "a"
  seeds:
\end{Verbatim}
\normalsize

An \emph{exit} of recursion class is 
a Choice expression with at least one alternative outside the class.
This alternative is a \emph{seed} that terminates the recursion.
An \emph{entry} to the class is the expression that starts left recursion.

\medskip
You may also want to find out how your grammar is affected by limited backtracking.
\newline
An interactive tool, the \textsl{PEG Explorer} assists you in checking the condition (*)
stated in Section~\ref{back}.
It is described in \url{http://mousepeg.sourceforge.net/explorer.htm}\,,
and is invoked as \tx{mouse.ExplorePEG}.




