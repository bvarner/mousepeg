%HHHHHHHHHHHHHHHHHHHHHHHHHHHHHHHHHHHHHHHHHHHHHHHHHHHHHHHHHHHHHHHHHHHHHHHHHH

\section{Left Recursion}\label{LeftRec}

%HHHHHHHHHHHHHHHHHHHHHHHHHHHHHHHHHHHHHHHHHHHHHHHHHHHHHHHHHHHHHHHHHHHHHHHHHH

Looking at specifications of programming langauges
you probably noticed that syntax of artithmetic expressions 
is often defined in a different way.
A sum of integers, such as in Section~\ref{TheGram},
may be defined like this:
\smallskip
\small
\begin{Verbatim}[frame=single,framesep=2mm,samepage=true,xleftmargin=15mm,xrightmargin=15mm,baselinestretch=0.8]
   Sum    = Sum "+" Number / Number ;
   Number = [0-9]+ ;
\end{Verbatim}
\normalsize
But this does not work as PEG. 
Procedure \Sum\ would invoke itself repeatedly on the same input, in an infinite descent.
We say that \Sum\ is \emph{left-recursive}.

The above, on the other hand, makes perfect sense when read as EBNF,
and defines \Sum\ as one or more \Number\,s separated by \tx{"+"}.

Beginning with Version 2.0, \Mouse\ accepts such "pseudo-PEG" expressions
and treats them as EBNF syntax definitions.
The way of defining semantics is unchanged.
The left-recursive version of example from Section~\ref{Semantics} may appear like this:

\smallskip 
\small
\begin{Verbatim}[frame=single,framesep=2mm,samepage=true,xleftmargin=15mm,xrightmargin=15mm,baselinestretch=0.8]
   Compute = Sum !_ {print} ;
   Sum     = Sum "+" Number {sum}
           / Number {pass} ;
   Number  = [0-9]+ {number} ;
\end{Verbatim}
\normalsize

It was necessary to add \Compute\ because you do not want to print the result
after each occurrence of \Sum.
The semantic action \Numbera\ is the same as in Section~\ref{Semantics}.
Action \tx{pass()} just passes the value from \Number\ to \Sum:

\smallskip 
\small
\begin{Verbatim}[frame=single,framesep=2mm,samepage=true,xleftmargin=15mm,xrightmargin=15mm,baselinestretch=0.8]
  //-------------------------------------------------------------------
  //  Sum = Number
  //          0
  //-------------------------------------------------------------------
  void pass()
    { lhs().put(rhs(0).get()); }
\end{Verbatim}
\normalsize

\newpage
Action \Suma\ does the actual computation
and \Printa\ prints the result.

\smallskip 
\small
\begin{Verbatim}[frame=single,framesep=2mm,samepage=true,xleftmargin=15mm,xrightmargin=15mm,baselinestretch=0.8]
  //-------------------------------------------------------------------
  //  Sum = Sum "+" Number
  //         0   1    2
  //-------------------------------------------------------------------
  void sum()
    { lhs().put((Integer)rhs(0).get() + (Integer)rhs(2).get()); }
\end{Verbatim}
\normalsize

You find the grammar and semantics file in \tx{example2R};
copy it to the \tx{work} directory.
Generating the parser may look like this:

\small
\begin{Verbatim}[samepage=true,xleftmargin=15mm,baselinestretch=0.8]
java mouse.Generate -P myParser -G myGrammar.txt -S mySemantics
Parsing procedures:
2 rules
0 inner
3 terminals
4 procedures for 1 left-recursion class(es)
\end{Verbatim}
\normalsize

You may then try the generated parser:
  
\small
\begin{Verbatim}[samepage=true,xleftmargin=15mm,baselinestretch=0.8]
java mouse.TryParser -P myParser
> 17+4711
4728
> 12+2+3
17
>
\end{Verbatim}
\normalsize

\Mouse\ does this trick by generating behind the scenes
a Parsing Expression Grammar that is equivalent to your PEG
and invokes the semantic actions in the correct context.
Looking at the generated parser, you can see that left-recursive expressions
have special procedures whose names begin with "\tx{\$}".

The generated grammar is not left-recursive 
under condition that your PEG does not contain "cycles".
A cycle means that procedure can call itself without consuming any input.
Here is an example: 

\smallskip 
\small
\begin{Verbatim}[frame=single,framesep=2mm,samepage=true,xleftmargin=15mm,xrightmargin=15mm,baselinestretch=0.8]
   A = B "a"? / "a" ;
   B = A "b"? / "b" ;
\end{Verbatim}
\normalsize

as \tx{"a"?} and \tx{"b"?} can succeed without consuming anything,
\tx{A} can call \tx{B}
and then call \tx{A} with nothing being consumed.

The support for left recursion is restricted to Choice and Sequence expressions.
In addition, the first expression in recursive Sequence must never accept an empty string.
The semantic action "on failure" (described later) and Boolean action (see \ref{BoolAct})
are not suported for left-recursive rules.

In directory \tx{example4R} you find the grammar of Example~4 from Section~\ref{Floating}
rewritten in left-recursive way:
\smallskip 
\small
\begin{Verbatim}[frame=single,framesep=2mm,samepage=true,xleftmargin=15mm,xrightmargin=15mm,baselinestretch=0.8]
   Compute = Space Sum !_ {print} ; 
   Sum     = Sum AddOp Number {sum}
           / Number {pass} ;
   Number  = Sign Digits? "." Digits Space {fraction}
           / Sign Digits Space {integer} ; 
   Sign    = ("-" Space)? ;
   AddOp   = [-+] Space ;
   Digits  = [0-9]+ ;
   Space   = " "* ;
\end{Verbatim}
\normalsize

As in the preceding case, it was necessary to add \Compute.
\Sign\ had to be moved to \Number\
so that it applies only to the first term of the sum.
You find semantic actions in the same directory.

Directory \tx{example8R} contains left-recursive version of Example~8 from Section~\ref{errors}. 

Directory \tx{example11} contains examples of left-recursive grammars found in the literature.




