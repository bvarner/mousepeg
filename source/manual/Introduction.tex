%HHHHHHHHHHHHHHHHHHHHHHHHHHHHHHHHHHHHHHHHHHHHHHHHHHHHHHHHHHHHHHHHHHHHHHHHHH

\section{Introduction}

%HHHHHHHHHHHHHHHHHHHHHHHHHHHHHHHHHHHHHHHHHHHHHHHHHHHHHHHHHHHHHHHHHHHHHHHHHH

Parsing Expression Grammar (PEG),
introduced by Ford in \cite{Ford:2004},
is a way to define the syntax of a programming language.
It encodes a recursive-descent parser for that language.
\Mouse\ is a tool to transcribe PEG into an executable parser.
Both \Mouse\ and the resulting parser are written in Java,
which makes them operating-system independent.
An integral feature of \Mouse\ is the mechanism for specifying
semantics (also in Java).

%---------------------------------------------------------------------- 
\subsubsection*{Recursive-descent parsing with backtracking}
%---------------------------------------------------------------------- 

Recursive-descent parsers consist of procedures that correspond to syntax rules.
The procedures call each other recursively, each being responsible for recognizing
input strings defined by its rule.
The syntax definition and its parser can be then viewed as the same thing,
presented in two different ways.
This design is very transparent and easy to modify when the language evolves.
The idea can be traced back to Lucas \cite{Lucas:1961}
who suggested it for grammars that later became known as the
Extended Backus-Naur Form (EBNF).

The problem is that procedures corresponding to certain types of syntax rules must decide
which procedure to call next.
It is easily solved for a class of languages that have the so-called LL(1) property:
the decision can be made by looking at the next input symbol.
Some syntax defintions can be transformed to satisfy this property.
However, forcing the language into the LL(1) mold
can make the grammar -- and the parser -- unreadable.

One can always use brute force: trial and error.
It means trying the alternatives one after another and backtracking after a failure.
But, an exhaustive search
may require exponential time,
so a possible option is \emph{limited} backtracking: never return after a partial success.
This approach has been used in actual parsers \cite{Brooker:Morris:1962,McClure:1965} 
and is described in \cite{Birman:1970,Birman:Ullman:1973,Hopgood:1969,Aho:Ullman:1972}.
It has been eventually formalized by Ford \cite{Ford:2004} under the name of Parsing Expression Grammar (PEG).

Wikipedia has an article on PEG \cite{Wiki:PEG},
and a dedicated Web site \cite{PEG} contains 
list of publications about PEGs and a link to discussion forum.

%---------------------------------------------------------------------- 
\subsubsection*{PEG programming}
%---------------------------------------------------------------------- 

Parsers defined by PEG
do not require a separate "lexer" or "scanner".
Together with lifting of the LL(1) restriction,
this gives a very convenient tool when you need
an ad-hoc parser for some application.
However, the limitation of backtracking may have unexpected effects
that give an impression of PEG being unpredictable.

In its external appearance, PEG is very like an EBNF grammar.
For some grammars, the limited backtracking is "efficient",
in the sense that it finds everything that would be found by full backtracking.
Parsing Expresion Grammar with efficient backtracking 
defines exactly the same language as its EBNF look-alike.
It agrees well with intuition
and is therefore easier to understand.
Some conditions for efficient backtracking have been identified 
in \cite{Redz:2014:FI}.
The \Mouse\ package includes a tool, the \textsl{PEG Explorer},
that assists you in checking whether your PEG has efficient backtracking.
You find its documentation at \url{mousepeg.sourceforge.net/explorer.htm}.

PEG has a feature that does not have an EBNF counterpart:
the syntactic predicates \tx{!e} and \tx{\&e}.
Not being part of EBNF, they have to be understood on their own.

%---------------------------------------------------------------------- 
\subsubsection*{Mouse - not a pack rat}
%---------------------------------------------------------------------- 

Even the limited backtracking may require a lot of time.
In \cite{Ford:2002}, PEG was introduced together with
a technique called \emph{packrat parsing}.
Packrat parsing handles backtracking
by extensive \emph{memoization}: storing all results
of parsing procedures.
It guarantees linear parsing time at a large memory cost.
(The name "packrat" comes from \emph{pack rat} -- a small rodent \emph{Neotoma cinerea}  
known for hoarding unnecessary items.)

Wikipedia \cite{Wiki:Rats} lists a number of generators  
producing packrat parsers from PEG.

The amount of backtracking does not matter 
in small interactive applications
where the input is short and performance not critical.
Moreover, the usual programming languages
do not require much backtracking.
%
Experiments reported in \cite{Redz:2007:FI,Redz:2008:FI}
demonstrated a moderate backtracking activity  
in PEG parsers for programming languages Java and~C.

\newpage
In view of these facts,
it is useful to construct PEG parsers
where the complexity of packrat technology is abandoned in favor
of simple and transparent design.
This is the idea of \Mouse:
a parser generator that transcribes
PEG into a set of recursive procedures that closely follow the grammar.
The name \Mouse\ was chosen in contrast to \textsl{Rats!}, 
one of the first generators producing packrat parsers \cite{Grimm:2004}.
Optionally, \Mouse\ can offer a small amount of memoization using the technique
described in \cite{Redz:2007:FI}.

%---------------------------------------------------------------------- 
\subsubsection*{Left recursion}
%---------------------------------------------------------------------- 

The recursive-descent process used by PEG parsers can not be applied to a grammar
that contains left recursion, because rule such as \tx{A = A"a"/"a"}
would result in an infinite descent.
But, for many reasons, this pattern is often used for defining the syntax
of programming languages.

Converting left recursion to iteration is possible, but is tedious, error-prone, 
and obscures the spirit of the grammar. 
Several ways of extending PEG to handle left recursion have been suggested 
\cite{Orlando:2010,Warth:2008,Tratt:2010,Medeiros:2014:LR}.

Version 2.0, \Mouse\ supports left recursion
using an experimental method of \emph{recursive ascent},
which continues the idea from \cite{Orlando:2010}.
It boils down to constructing behind the scenes an alternative PEG
that takes care of left-recursive portions.
It is explained in \url{mousepeg.sourceforge.net/Documents/RecursiveAscent.pdf}.
The support is limited to the Choice and Sequence expressions,
and imposes some restrictions on expressions and semantic actions. 

\bigskip
After a short presentation of PEG in the following section,
sections \ref{GetStarted} through \ref{LeftRec}
have the form of a tutorial,
introducing the reader to \Mouse\ by hands-on experience.
They are followed by descriptions of some extensions
and details not covered in the tutorial.
