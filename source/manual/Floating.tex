\newpage
%HHHHHHHHHHHHHHHHHHHHHHHHHHHHHHHHHHHHHHHHHHHHHHHHHHHHHHHHHHHHHHHHHHHHHHHHHH

\section{Let's go floating}\label{Floating}

%HHHHHHHHHHHHHHHHHHHHHHHHHHHHHHHHHHHHHHHHHHHHHHHHHHHHHHHHHHHHHHHHHHHHHHHHHH

To illustrate some more features of \Mouse,
we shall add floating-point numbers to your primitive calculator.
The calculations will now be done on numbers in \tx{double} form.
The only thing needed in the grammar is the possibility to specify such numbers.
We decide for a format with one or more digits after the decimal point,
and optional digits before: \tx{Digits?~"."~Digits}.
We are going to keep the integer format as an alternative, 
which means defining \Number\ as

\small
\begin{Verbatim}[samepage=true,xleftmargin=15mm,baselinestretch=0.8]
         Number = Digits? "." Digits Space / Digits Space ;
\end{Verbatim}
\normalsize

The two alternatives produce two quite different configurations of the "right-hand side".
The semantic action would have to begin by identifying them.
But, this has already been done by the parser.
To exploit that, you specify a separate semantic action
for each alternative in a rule that defines choice expression.
It means modifying your grammar as follows:
%
\small
\begin{Verbatim}[frame=single,framesep=2mm,samepage=true,xleftmargin=15mm,xrightmargin=15mm,baselinestretch=0.8]
   Sum     = Space Sign Number (AddOp Number)* !_ {sum} ;
   Number  = Digits? "." Digits Space {fraction}
           / Digits Space {integer} ; 
   Sign    = ("-" Space)? ;
   AddOp   = [-+] Space ;
   Digits  = [0-9]+ ;
   Space   = " "* ;
\end{Verbatim}
\normalsize
%
You specify here two semantic actions, \tx{fraction()} and \tx{integer()}, 
for the two alternatives of \Number.

The first of them is invoked to handle input of the form \tx{Digits?~"."~Digits~Space}.
According to the rule given in Section~\ref{RHS}, it can result
in two possible configurations of the "right-hand side":
 
\small
\begin{Verbatim}[samepage=true,xleftmargin=15mm,baselinestretch=0.9]
         Digits "." Digits Space       or       "." Digits Space
            0    1     2     3                   0     1     2
\end{Verbatim}
\normalsize
 
You need to convert the text that represents the floating-point number
into a \tx{Double} object.
You can obtain this text
by using the helper method \tx{rhsText($i,j$)}.
The method returns the text identified by the right-hand side
elements $i$ through $j-1$.
As you can easily see, \tx{rhsText(0,rhsSize()-1)} will do the
job for each of the two configurations, so \tx{fraction()} can be coded as follows:
 
\small
\begin{Verbatim}[frame=single,framesep=2mm,samepage=true,xleftmargin=15mm,xrightmargin=15mm,baselinestretch=0.8]
   //-------------------------------------------------------------------
   //  Number = Digits? "."  Digits Space
   //              0    1(0)  2(1)   3(2)
   //-------------------------------------------------------------------
   void fraction()
     { lhs().put(Double.valueOf(rhsText(0,rhsSize()-1))); }
\end{Verbatim}
\normalsize
 
As computation will now be done in floating point,
\tx{integer()} should also return a \tx{Double}:
 
\small
\begin{Verbatim}[frame=single,framesep=2mm,samepage=true,xleftmargin=15mm,xrightmargin=15mm,baselinestretch=0.8]
   //-------------------------------------------------------------------
   //  Number = Digits Space
   //              0     1
   //-------------------------------------------------------------------
   void integer()
     { lhs().put(Double.valueOf(rhs(0).text())); }
\end{Verbatim}
\normalsize

The semantic action for \Sum\ remains unchanged, except 
that \tx{s} becomes \tx{double} 
and the casts are to \tx{Double} instead of \tx{Integer}.

You may copy the new grammar and semantics from \tx{example4},
generate new parser (replacing the old), and compile both.
A session with the new parser may look like this:

\small
\begin{Verbatim}[samepage=true,xleftmargin=15mm,baselinestretch=0.8]
 java mouse.TryParser -P myParser
 > -123 + 456
 333.0
  > 1.23 - 4.56
 -3.3299999999999996
 > 
\end{Verbatim}
\normalsize
%
(Who said floating-point arithmetic is exact?)





