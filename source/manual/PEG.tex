%HHHHHHHHHHHHHHHHHHHHHHHHHHHHHHHHHHHHHHHHHHHHHHHHHHHHHHHHHHHHHHHHHHHHHHHHHH

\section{Parsing Expression Grammar}

%HHHHHHHHHHHHHHHHHHHHHHHHHHHHHHHHHHHHHHHHHHHHHHHHHHHHHHHHHHHHHHHHHHHHHHHHHH
%----------------------------------------------------------------------
\subsection{Parsing expressions\label{PEG}}
%----------------------------------------------------------------------

Parsing expressions are instructions for parsing strings.
You can think of a parsing expression as shorthand for a procedure 
that carries out such instruction.

Parsing expression is applied to input text -- a character string --
and tries to recognize the initial portion of that text.
If it succeeds, it "consumes" the recognized portion
and indicates "success"; 
otherwise, it indicates "failure" 
and does not consume anything.

The following five kinds of parsing expressions work directly
on the input text:

\medskip
\begin{tabular}{|c|p{0.86\textwidth}|}
\hline
$\tx{"}s\tx{"}$\upsp
   & where $s$ is a nonempty character string.
     If the text starts with the string $s$,
     consume that string and indicate success.
     Otherwise indicate failure.\dnsp \\
\hline
$\tx{[}$s$\tx{]}$\upsp 
   & where $s$ is a nonempty character string.
     If the text starts with a character appearing in $s$,
     consume that character and indicate success.
     Otherwise indicate failure.\dnsp \\    
\hline
$\tx{\textasciicircum[}$s$\tx{]}$\upsp 
   & where $s$ is a nonempty character string.
     If the text starts with a character \emph{not} appearing in $s$,
     consume that character and indicate success.
     Otherwise indicate failure.\dnsp \\    
\hline
$\tx{[}c_1\tx{-}\,c_2\tx{]}$\upsp
   & where $c_1,c_2$ are two characters.
     If the text starts with a character from the range 
     $c_1$~through~$c_2$, consume that character and indicate success.
     Otherwise indicate failure.\dnsp \\
\hline
\tx{\_}\upsp
   & where \tx{\_\,} is the underscore character.
     If there is a character ahead, consume it and indicate success.
     Otherwise (that is, at the end of input) indicate failure.\dnsp\\
\hline
\end{tabular}

\medskip
These expressions are analogous to terminals of a classical context-free
grammar, and are referred to as "terminals".
The remaining kinds of parsing expressions
invoke other expressions to do their job:

\medskip
\begin{tabular}{|c|p{0.83\linewidth}|}
\hline
$e\tx{?}$\upsp
   & Invoke the expression $e$ and indicate success whether it succeeded or not.\dnsp\\
\hline
$e\tx{*}$\upsp
   & Invoke the expression $e$ repeatedly as long as it succeeds.\newline
     Indicate success even if it did not succeed a single time.\dnsp\\
\hline
$e\tx{+}$\upsp
   & Invoke the expression $e$ repeatedly as long as it succeeds.\newline
     Indicate success if $e$ succeeded at least once.
     Otherwise indicate failure.\dnsp\\
\hline
$\tx{\&}e$\upsp
   & Invoke the expression $e$ and then
     reset the input as it was before the invocation of $e$\newline
     (do not consume anything).
     Indicate success if $e$ succeeded or failure if $e$ failed.\dnsp\\
\hline
$\tx{!}e$\upsp
   & Invoke the expression $e$ and then
     reset the input as it was before the invocation of $e$\newline
     (do not consume anything).
     Indicate success if $e$ failed or failure if $e$ succeeded.\dnsp\\     
\hline
$e_1 \ldots e_n$\upsp
   & Invoke expressions $e_1,\ldots,e_n$, in this order,
     as long as they succeed.\newline
     Indicate success if all succeeded. 
     Otherwise reset input as it was before the invocation of $e_1$\newline
     (do not consume anything) and indicate failure.\dnsp\\
\hline
$e_1\,\tx{/}\ldots\,\tx{/}\,e_n$\upsp
   & Invoke expressions $e_1,\ldots,e_n$, in this order,
     until one of them succeeds.\newline 
     Indicate success if one of expressions succeeded.
     Otherwise indicate failure.\dnsp\\
\hline
\end{tabular}

\medskip
The expressions $e, e_1, \ldots, e_n$ above can be specified either explicitly
or by name (the way of naming expressions will be explained in a short while).
An expression specified explicitly within another expression
with the same or higher precedence (see below)
must be enclosed in parentheses.
The following table summarizes all forms of parsing expressions.

% ---------------------------------------------------------------------
\begin{center}
\begin{tabular}{|c|l|c|} \hline
expression & name & precedence\upsp\dnsp \\ \hline
$\tx{"}s\tx{"}$  & String Literal \upsp & 5 \\
$\tx{[}s\tx{]}$  & Character Class & 5 \\
$\tx{\textasciicircum[}s\tx{]}$  & Not Character Class & 5 \\
$\tx{[}c_1\tx{-}\,c_2\tx{]}$ & Character Range & 5 \\
\textbf{\_} & Any Character & 5 \\
$e\tx{?}$  & Optional & 4 \\
$e\tx{*}$  & Iterate & 4 \\
$e\tx{+}$  & One or More & 4 \\
$\tx{\&}e$ & Ahead & 3 \\
$\tx{!}e$  & Not Ahead & 3 \\
$e_1 \ldots e_n$  & Sequence & 2 \\
$e_1\,\tx{/}\ldots\,\tx{/}\,e_n$  & Choice\dnsp & 1 \\ \hline
\end{tabular}
\end{center}
% ---------------------------------------------------------------------

\medskip
Backtracking takes place in the Sequence expression.
If $e_1,\ldots,e_i$ in $e_1\ldots e_i\ldots e_n$ succeed and consumed some input,
but then $e_{i\,+1}$ fails, the input is reset as it was before trying $e_1$;
the whole expression fails.
If this expression was invoked from a Choice expression (and was not the last there),
Choice has an opportunity to try another alternative on the same input.

However, the opportunities to try another alternative are limited:
once $e_i$ in the Choice \tx{$e_1/\ldots/e_i/\ldots /e_n$} succeeded, 
none of the alternatives $e_{i\,+1},\ldots e_n$ will ever be tried on the same input,
even if the parse fails later on.
This is the limited backtracking.

%----------------------------------------------------------------------
\subsection{The grammar\label{TheGram}}
%----------------------------------------------------------------------

Parsing Expression Grammar is a list of one or more "rules" of the form:
%
\begin{equation*}
\textit{name}\quad \tx{=}\quad \textit{expr} \;\;\tx{;}
\end{equation*}
%
where \textit{expr} is a parsing expression,
and \textit{name} is a name given to it.
The \textit{name} is a string of one or more letters (\tx{a-z}, \tx{A-Z}) and/or digits,
starting with a letter. 
White space is allowed everywhere except inside names.
Comments starting with a double slash and extending to the end of a line are also allowed.

The order of the rules does not matter, except that the expression specified first
is the "start expression", invoked at the start of the parser.

\medskip
A specific grammar may look like this:

\smallskip
\small
\begin{Verbatim}[frame=single,framesep=2mm,samepage=true,xleftmargin=15mm,xrightmargin=15mm,baselinestretch=0.8]
   Sum    = Number ("+" Number)* !_ ;
   Number = [0-9]+ ;
\end{Verbatim}
\normalsize
%
It consists of two named expressions: \tx{Sum} and \tx{Number}.
They define a parser consisting of
two procedures named \tx{Sum} and \tx{Number}.
The parser starts by invoking \tx{Sum}.
The \tx{Sum} invokes \tx{Number}, and if this succeeds,
repeatedly invokes \tx{("+" Number)} as long as it succeeds.
Finally, \tx{Sum}
invokes a sub-procedure for the predicate "\verb#!_#",
which succeeds only if it does not see any character ahead --
that is, only at the end of input.
The \tx{Number} reads digits in the range from 
0 through 9 as long as it succeeds,
and is expected to find at least one such digit.
%
One can easily see that the parser
accepts strings like "\tx{2+2}",  "\tx{17+4711}", or "\tx{2}";
in general, one or more integers separated by "\tx{+}".

It is quite obvious that
the expression $e$ in $e\tx{*}$ and $e\tx{+}$ must never succeed
in consuming an empty string;
otherwise, the parser could run into an infinite loop.
\Mouse\ checks this and gives you a warning if needed.

