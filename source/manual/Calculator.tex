%HHHHHHHHHHHHHHHHHHHHHHHHHHHHHHHHHHHHHHHHHHHHHHHHHHHHHHHHHHHHHHHHHHHHHHHHHH

\section{Calculator with memory\label{calc}}

%HHHHHHHHHHHHHHHHHHHHHHHHHHHHHHHHHHHHHHHHHHHHHHHHHHHHHHHHHHHHHHHHHHHHHHHHHH

Back to the calculator from Section~\ref{arith}.
We suggest one more improvement:
the ability to store computed values for subsequent use.
The idea is that  
a session may look like this:

\small
\begin{Verbatim}[samepage=true,xleftmargin=15mm,baselinestretch=0.8]
 java Calc
 > n = 3/2
 > n = (3/n + n)/2
 > (3/n + n)/2
   1.7321428571428572
\end{Verbatim}
\normalsize

In the first line we store the number \tx{3/2} under name "\tx{n}".
In the second, we replace it by \tx{(3/n + n)/2}, computed using the stored value.
Finally, we use the newest \tx{n} to compute one more value and print it.
(Is it just an accident that the result resembles $\sqrt{3}$\,?)

\newpage
To achieve this kind of functionality, you can modify the grammar as follows:

\small
\begin{Verbatim}[frame=single,framesep=2mm,samepage=true,xleftmargin=15mm,xrightmargin=15mm,baselinestretch=0.8]
   Input   = Space (Store / Print) !_ ;
   Store   = Name Space Equ Sum {store} ;
   Print   = Sum {print} ;
   Sum     = Sign Product (AddOp Product)* {sum} ;
   Product = Factor (MultOp Factor)* {product} ;
   Factor  = Digits? "." Digits Space {fraction}
           / Digits Space {integer}
           / Lparen Sum Rparen {unwrap} 
           / Name Space {retrieve} ; 
   Sign    = ("-" Space)? ;
   AddOp   = [-+] Space ;
   MultOp  = ([*/] Space)? ;
   Lparen  = "("  Space ;
   Rparen  = ")"  Space ;
   Equ     = "="  Space ;
   Digits  = [0-9]+ ;
   Name    = [a-z]+ ;
   Space   = " "* ;
\end{Verbatim}
\normalsize

We have here two alternative forms of \tx{Input}
to perform the two tasks,
and one more way to obtain a \tx{Factor}: 
by retrieving a stored result.
The results will be stored in a hash table \tx{dictionary}
that resides in the semantics object.

The semantic actions for \tx{Store} and \tx{Print} are:

\small
\begin{Verbatim}[frame=single,framesep=2mm,samepage=true,xleftmargin=15mm,xrightmargin=15mm,baselinestretch=0.8]
   //-------------------------------------------------------------------
   //  Store = Name Space Equ Sum
   //            0    1    2   3
   //-------------------------------------------------------------------
   void store()
     { dictionary.put(rhs(0).text(),(Double)rhs(3).get()); }
\end{Verbatim}
\normalsize

\small
\begin{Verbatim}[frame=single,framesep=2mm,samepage=true,xleftmargin=15mm,xrightmargin=15mm,baselinestretch=0.8]
   //-------------------------------------------------------------------
   //  Print = Sum
   //           0
   //-------------------------------------------------------------------
   void print()
     { System.out.println((Double)rhs(0).get()); }
\end{Verbatim}
\normalsize

The stored result is retrieved by semantic action
for the fourth alternative of \tx{Factor}:

\small
\begin{Verbatim}[frame=single,framesep=2mm,samepage=true,xleftmargin=15mm,xrightmargin=15mm,baselinestretch=0.8]
   //-------------------------------------------------------------------
   //  Factor = Name Space
   //             0    1
   //-------------------------------------------------------------------
   void retrieve()
     {
       String name = rhs(0).text();
       Double d = dictionary.get(name);
       if (d==null)
       {
         d = (Double)Double.NaN;
         System.out.println
           (rhs(0).where(0) + ": '" + name + "' is not defined.");
       }
       lhs().put(d);
     }
\end{Verbatim}
\normalsize

You have to handle the case where the name is not in the dictionary.
The example shown here returns in such a case a \tx{NaN} (Not a Number) object. 
Using \tx{NaN} in subsequent calculations produces inevitably a \tx{NaN} result.
A message printed to \tx{System.out} explains where this \tx{NaN} comes from.
Note the call to \tx{rhs(0).where(0)} that returns a text telling where to find the name
in the input. 

You find the modified grammar and semantics in \tx{example7}.
You find there also a file \tx{Calc.java} containing invocation of the calculator.
Copy these files to \tx{work} directory, generate a new parser, and compile
everything.
You can now invoke the calculator as shown at the beginning of this section.
