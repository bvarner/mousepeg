\documentclass[a4paper,fleqn]{article}
%
\usepackage{amsmath}
\usepackage{amsfonts}
\usepackage{amssymb}
\usepackage{amsthm}
\usepackage{cite}
\usepackage{calc}
\usepackage{array}
\usepackage{texdraw}
\usepackage{fancyvrb}
\usepackage{url}
\usepackage{xcolor}
\usepackage[dvipdfm]{hyperref}
\hypersetup
{ colorlinks,
  linkcolor={blue},
  citecolor={blue},
  urlcolor={blue}
}
\usepackage{titletoc}

    \titlecontents{section}[0em]{\smallskip}
    {\thecontentslabel\enspace}%numbered sections
    {}%numberless section
    {\titlerule*[1.5pc]{}\contentspage}
\usepackage[nottoc,numbib]{tocbibind}    
\paperwidth 210mm
\paperheight 297mm
\textwidth 160mm
\textheight 240mm
\hoffset 0mm
\voffset 0mm
\evensidemargin 0mm
\oddsidemargin 0mm
\topmargin 0mm
\headheight 0mm
\headsep 0mm
\parindent 0em
\parskip 0.5em
\footnotesep 1em
\setlength{\skip\footins}{6mm}

\setcounter{tocdepth}{1}

%========================================================================
%  Shorthands
%========================================================================
\newcommand{\Version}{2.3}

\newcommand{\Compute}{\tx{Compute}}
\newcommand{\Digits}{\tx{Digits}}
\newcommand{\Digitsa}{\tx{digits()}}
\newcommand{\Digitsb}{\tx{Digits()}}
\newcommand{\Factor}{\tx{Factor}}
\newcommand{\Factorb}{\tx{Factor()}}
\newcommand{\Input}{\tx{Input}}
\newcommand{\Inputa}{\tx{input}}
\newcommand{\Inputb}{\tx{Input()}}
\newcommand{\Mouse}{\textsl{Mouse}}
\newcommand{\Number}{\tx{Number}}
\newcommand{\Numbera}{\tx{number()}}
\newcommand{\Numberb}{\tx{Number()}}
\newcommand{\Object}{\tx{Object}}
\newcommand{\Phrase}{\tx{Phrase}}
\newcommand{\Plus}{\tx{Plus}}
\newcommand{\Plusa}{\tx{plus()}}
\newcommand{\Plusb}{\tx{Plus()}}
\newcommand{\Print}{\tx{Print}}
\newcommand{\Printa}{\tx{print()}}
\newcommand{\Printb}{\tx{Print()}}
\newcommand{\Product}{\tx{Product}}
\newcommand{\Producta}{\tx{product()}}
\newcommand{\Productb}{\tx{Product()}}
\newcommand{\Sign}{\tx{Sign}}
\newcommand{\Signb}{\tx{Sign()}}
\newcommand{\Space}{\tx{Space}}
\newcommand{\Spacea}{\tx{space()}}
\newcommand{\Spaceb}{\tx{Space()}}
\newcommand{\Store}{\tx{Store}}
\newcommand{\Storea}{\tx{store()}}
\newcommand{\Storeb}{\tx{Store()}}
\newcommand{\String}{\tx{String}}
\newcommand{\Sum}{\tx{Sum}}
\newcommand{\Suma}{\tx{sum()}}
\newcommand{\Sumb}{\tx{Sum()}}

\newcommand{\follow}{\operatorname{follow}}
\newcommand{\Tail}{\operatorname{Tail}}
\newcommand{\sL}{\mathcal{L}}             % Script L
\newcommand{\Pref}{\operatorname{Pref}}
\renewcommand{\emptyset}{\varnothing}     % Empty set

%========================================================================
%  Spacing in tables
%========================================================================
\newcommand{\dnsp}{\rule[-1.4ex]{0ex}{1ex}}    % Space below text
\newcommand{\upsp}{\rule{0ex}{2.9ex}}          % Space above text
\newcommand{\prop}{\rule[-0.4ex]{0ex}{2.5ex}}  % Space in boxes

%========================================================================
%  TT font
%========================================================================
\newcommand{\stx}[1]{\small\texttt{#1}\normalsize}
\newcommand{\tx}[1]{\texttt{#1}}
 
%========================================================================
%  Unordered list
%========================================================================
\newcommand{\ul}
{\begin{list}
{--}
 {\setlength{\topsep}{0.5ex}
  \setlength{\itemsep}{0ex}
  \setlength{\parsep}{0ex}
  \setlength{\itemindent}{0em}
  \setlength{\labelwidth}{1em}
  \setlength{\labelsep}{0.5em}
  \setlength{\leftmargin}{1.5em}
 }
}
\newcommand{\eul}{\end{list}}
 
%========================================================================
%  entry
%========================================================================
\newcommand{\entrylabel}[1]{{#1}\dnsp\hfill}
\newenvironment{entry}
  {\begin{list}{}%
     {\renewcommand{\makelabel}{\entrylabel}%
       \setlength{\labelwidth}{10pt}%
       \setlength{\leftmargin}{\labelwidth+\labelsep}%
       \setlength{\itemsep}{12pt}%
     }%
  }%
  {\end{list}}

%========================================================================
%  Texdraw macros
%========================================================================
\newcommand{\phrase}[6] % #1=class #2=position #3=value #4=top text #5=bottom text #6=ref
  { 
    \linewd 0.1
    \textref h:C v:C
    \savecurrpos(*#6cx *#6cy)                           % Save lower left corner
    \rmove (12 15) \savecurrpos (*#6tx *#6ty)           % Save mid-points of box sides
    \rmove (0 -15) \savecurrpos (*#6bx *#6by)
    \move(*#6cx *#6cy)
    \rmove (0 7.5) \savecurrpos (*#6lx *#6ly)
    \rmove (24 0)  \savecurrpos (*#6rx *#6ry)
    \move(*#6cx *#6cy)                                  % Draw box
    \rlvec (24 0) \rlvec (0 15) 
    \rlvec (-24 0) \rlvec (0 -15)
    \move(*#6cx *#6cy) \rmove (0 5) \rlvec(24 0)        % .. with line inside
    \move(*#6cx *#6cy) \rmove (12 12)   \htext{\tx{#1}} % Class name
    \move(*#6cx *#6cy) \rmove (12 7.5)  \htext{\tx{#2}} % Consumed text
    \move(*#6cx *#6cy) \rmove (12 2.2)  \htext{\tx{#3}} % Semantic value
    \move(*#6cx *#6cy) \rmove (12 18) \htext{\tx{#4}}   % Text above
    \move(*#6cx *#6cy)
  }
  
\newcommand{\lowphrase}[6] % #1=class #2=position #3=value #4=no top text #5=no bottom text #6=ref
  { 
    \linewd 0.1
    \textref h:C v:C
    \savecurrpos(*#6cx *#6cy)                           % Save lower left corner
    \rmove (12 15) \savecurrpos (*#6tx *#6ty)           % Save mid-points of box sides
    \rmove (0 -15) \savecurrpos (*#6bx *#6by)
    \move(*#6cx *#6cy)
    \rmove (0 7.5) \savecurrpos (*#6lx *#6ly)
    \rmove (24 0)  \savecurrpos (*#6rx *#6ry)
    \move(*#6cx *#6cy)                                  % Draw box
    \rlvec (24 0) \rlvec (0 15) 
    \rlvec (-24 0) \rlvec (0 -15)
    \move(*#6cx *#6cy) \rmove (0 5) \rlvec(24 0)        % .. with line inside
    \move(*#6cx *#6cy) \rmove (12 12)   \htext{\tx{#1}} % Class name
    \move(*#6cx *#6cy) \rmove (12 7.5)  \htext{\tx{#2}} % Consumed text
    \move(*#6cx *#6cy) \rmove (12 2.2)  \htext{\tx{#3}} % Semantic value
    \move(*#6cx *#6cy)
  }
  
%HHHHHHHHHHHHHHHHHHHHHHHHHHHHHHHHHHHHHHHHHHHHHHHHHHHHHHHHHHHHHHHHHHHHHHHHHH

%  Title page

%HHHHHHHHHHHHHHHHHHHHHHHHHHHHHHHHHHHHHHHHHHHHHHHHHHHHHHHHHHHHHHHHHHHHHHHHHH
\begin{document}
\fontfamily{ptm}\selectfont

\pagestyle{empty}
\vspace*{\stretch{1}}
\begin{center}
\rule{\linewidth-20mm}{.5mm}

\bigskip
\Large \textbf{\textit{MOUSE}: FROM PARSING EXPRESSIONS\\TO A PRACTICAL PARSER}

\bigskip
\Large Version \Version

\bigskip
\Large Roman R. Redziejowski
\rule{\linewidth-20mm}{.5mm}
\vspace*{\stretch{1}}

%HHHHHHHHHHHHHHHHHHHHHHHHHHHHHHHHHHHHHHHHHHHHHHHHHHHHHHHHHHHHHHHHHHHHHHHHHH

%  Abstract

%HHHHHHHHHHHHHHHHHHHHHHHHHHHHHHHHHHHHHHHHHHHHHHHHHHHHHHHHHHHHHHHHHHHHHHHHHH

\normalsize
\parbox{0.875\linewidth}{
\noindent
Parsing Expression Grammar (PEG) is a way to specify
recursive-descent parsers with limited backtracking.
The use of backtracking lifts the $LL(1)$ restriction usually imposed
by top-down parsers.
In addition, PEG can define parsers with integrated lexing.

\medskip
\noindent
\Mouse\ is a tool to transcribe PEG into an executable parser written in Java.
Unlike some existing PEG generators (e.g., \textsl{Rats!}), \Mouse\
does not produce a storage-hungry "packrat parser",
but a collection of transparent recursive procedures.

\medskip
\noindent
An integral feature of \Mouse\ is the mechanism for specifying
semantics (also in Java).
This makes \Mouse\ a convenient tool if one needs an ad-hoc language processor.
Being written in Java, the processor is operating-system independent.

\medskip
\noindent
Starting with Version 2.0, \Mouse\ offers some support for left-recursive grammars.

\medskip
\noindent
This is a user's manual in the form of a tutorial
that introduces the reader to \Mouse\ through 
a hands-on experience.}
\end{center}

\vspace*{\stretch{3}}

\begin{center}
April 30, 2021

\end{center}
\newpage


\vspace*{\stretch{1}}
\noindent
Copyright \copyright\ 2009, 2010, 2011, 2012, 2013, 2014, 2015, 2016, 2017, 2019, 2020, 2021
\newline
by Roman R. Redziejowski (\tx{www.romanredz.se}).

\noindent
The author gives unlimited permission to copy, translate and/or distribute
this document, with or without modifications, 
as long as this notice is preserved,
and information is provided about any changes.

\noindent
This document is available at \url{mousepeg.sourceforge.net/Documents/Manual.pdf}.
\newpage
\tableofcontents

\newpage
\vspace*{\stretch{1}}
\subsection*{Changes from version of July 1, 2020:}
\ul
\item Added sections \ref{Extensions} and \ref{Checking}.
\item Removed second section on backtracking.
\item Introduced new operators "\tx{:}" and "\tx{:!}", described in Section~\ref{Extensions}.
\item Description of operators "\tx{*+}" and "\tx{++}" moved to Section~\ref{Extensions}.
\eul

\newpage
\pagestyle{plain}
\setcounter{page}{1}

%HHHHHHHHHHHHHHHHHHHHHHHHHHHHHHHHHHHHHHHHHHHHHHHHHHHHHHHHHHHHHHHHHHHHHHHHHH

\section{Introduction}

%HHHHHHHHHHHHHHHHHHHHHHHHHHHHHHHHHHHHHHHHHHHHHHHHHHHHHHHHHHHHHHHHHHHHHHHHHH

Parsing Expression Grammar (PEG),
introduced by Ford in \cite{Ford:2004},
is a way to define the syntax of a programming language.
It encodes a recursive-descent parser for that language.
\Mouse\ is a tool to transcribe PEG into an executable parser.
Both \Mouse\ and the resulting parser are written in Java,
which makes them operating-system independent.
An integral feature of \Mouse\ is the mechanism for specifying
semantics (also in Java).

%---------------------------------------------------------------------- 
\subsubsection*{Recursive-descent parsing with backtracking}
%---------------------------------------------------------------------- 

Recursive-descent parsers consist of procedures that correspond to syntax rules.
The procedures call each other recursively, each being responsible for recognizing
input strings defined by its rule.
The syntax definition and its parser can be then viewed as the same thing,
presented in two different ways.
This design is very transparent and easy to modify when the language evolves.
The idea can be traced back to Lucas \cite{Lucas:1961}
who suggested it for grammars that later became known as the
Extended Backus-Naur Form (EBNF).

The problem is that procedures corresponding to certain types of syntax rules must decide
which procedure to call next.
It is easily solved for a class of languages that have the so-called LL(1) property:
the decision can be made by looking at the next input symbol.
Some syntax defintions can be transformed to satisfy this property.
However, forcing the language into the LL(1) mold
can make the grammar -- and the parser -- unreadable.

One can always use brute force: trial and error.
It means trying the alternatives one after another and backtracking after a failure.
But, an exhaustive search
may require exponential time,
so a possible option is \emph{limited} backtracking: never return after a partial success.
This approach has been used in actual parsers \cite{Brooker:Morris:1962,McClure:1965} 
and is described in \cite{Birman:1970,Birman:Ullman:1973,Hopgood:1969,Aho:Ullman:1972}.
It has been eventually formalized by Ford \cite{Ford:2004} under the name of Parsing Expression Grammar (PEG).

Wikipedia has an article on PEG \cite{Wiki:PEG},
and a dedicated Web site \cite{PEG} contains 
list of publications about PEGs and a link to discussion forum.

%---------------------------------------------------------------------- 
\subsubsection*{PEG programming}
%---------------------------------------------------------------------- 

Parsers defined by PEG
do not require a separate "lexer" or "scanner".
Together with lifting of the LL(1) restriction,
this gives a very convenient tool when you need
an ad-hoc parser for some application.
However, the limitation of backtracking may have unexpected effects
that give an impression of PEG being unpredictable.

In its external appearance, PEG is very like an EBNF grammar.
For some grammars, the limited backtracking is "efficient",
in the sense that it finds everything that would be found by full backtracking.
Parsing Expresion Grammar with efficient backtracking 
defines exactly the same language as its EBNF look-alike.
It agrees well with intuition
and is therefore easier to understand.
Some conditions for efficient backtracking have been identified 
in \cite{Redz:2014:FI}.
The \Mouse\ package includes a tool, the \textsl{PEG Explorer},
that assists you in checking whether your PEG has efficient backtracking.
You find its documentation at \url{mousepeg.sourceforge.net/explorer.htm}.

PEG has a feature that does not have an EBNF counterpart:
the syntactic predicates \tx{!e} and \tx{\&e}.
Not being part of EBNF, they have to be understood on their own.

%---------------------------------------------------------------------- 
\subsubsection*{Mouse - not a pack rat}
%---------------------------------------------------------------------- 

Even the limited backtracking may require a lot of time.
In \cite{Ford:2002}, PEG was introduced together with
a technique called \emph{packrat parsing}.
Packrat parsing handles backtracking
by extensive \emph{memoization}: storing all results
of parsing procedures.
It guarantees linear parsing time at a large memory cost.
(The name "packrat" comes from \emph{pack rat} -- a small rodent \emph{Neotoma cinerea}  
known for hoarding unnecessary items.)

Wikipedia \cite{Wiki:Rats} lists a number of generators  
producing packrat parsers from PEG.

The amount of backtracking does not matter 
in small interactive applications
where the input is short and performance not critical.
Moreover, the usual programming languages
do not require much backtracking.
%
Experiments reported in \cite{Redz:2007:FI,Redz:2008:FI}
demonstrated a moderate backtracking activity  
in PEG parsers for programming languages Java and~C.

\newpage
In view of these facts,
it is useful to construct PEG parsers
where the complexity of packrat technology is abandoned in favor
of simple and transparent design.
This is the idea of \Mouse:
a parser generator that transcribes
PEG into a set of recursive procedures that closely follow the grammar.
The name \Mouse\ was chosen in contrast to \textsl{Rats!}, 
one of the first generators producing packrat parsers \cite{Grimm:2004}.
Optionally, \Mouse\ can offer a small amount of memoization using the technique
described in \cite{Redz:2007:FI}.

%---------------------------------------------------------------------- 
\subsubsection*{Left recursion}
%---------------------------------------------------------------------- 

The recursive-descent process used by PEG parsers can not be applied to a grammar
that contains left recursion, because rule such as \tx{A = A"a"/"a"}
would result in an infinite descent.
But, for many reasons, this pattern is often used for defining the syntax
of programming languages.

Converting left recursion to iteration is possible, but is tedious, error-prone, 
and obscures the spirit of the grammar. 
Several ways of extending PEG to handle left recursion have been suggested 
\cite{Orlando:2010,Warth:2008,Tratt:2010,Medeiros:2014:LR}.

Version 2.0, \Mouse\ supports left recursion
using an experimental method of \emph{recursive ascent},
which continues the idea from \cite{Orlando:2010}.
It boils down to constructing behind the scenes an alternative PEG
that takes care of left-recursive portions.
It is explained in \url{mousepeg.sourceforge.net/Documents/RecursiveAscent.pdf}.
The support is limited to the Choice and Sequence expressions,
and imposes some restrictions on expressions and semantic actions. 

\bigskip
After a short presentation of PEG in the following section,
sections \ref{GetStarted} through \ref{LeftRec}
have the form of a tutorial,
introducing the reader to \Mouse\ by hands-on experience.
They are followed by descriptions of some extensions
and details not covered in the tutorial.
 % 1 Introduction

%HHHHHHHHHHHHHHHHHHHHHHHHHHHHHHHHHHHHHHHHHHHHHHHHHHHHHHHHHHHHHHHHHHHHHHHHHH

\section{Parsing Expression Grammar}

%HHHHHHHHHHHHHHHHHHHHHHHHHHHHHHHHHHHHHHHHHHHHHHHHHHHHHHHHHHHHHHHHHHHHHHHHHH
%----------------------------------------------------------------------
\subsection{Parsing expressions\label{PEG}}
%----------------------------------------------------------------------

Parsing expressions are instructions for parsing strings.
You can think of a parsing expression as shorthand for a procedure 
that carries out such instruction.

Parsing expression is applied to input text -- a character string --
and tries to recognize the initial portion of that text.
If it succeeds, it "consumes" the recognized portion
and indicates "success"; 
otherwise, it indicates "failure" 
and does not consume anything.

The following five kinds of parsing expressions work directly
on the input text:

\medskip
\begin{tabular}{|c|p{0.86\textwidth}|}
\hline
$\tx{"}s\tx{"}$\upsp
   & where $s$ is a nonempty character string.
     If the text starts with the string $s$,
     consume that string and indicate success.
     Otherwise indicate failure.\dnsp \\
\hline
$\tx{[}$s$\tx{]}$\upsp 
   & where $s$ is a nonempty character string.
     If the text starts with a character appearing in $s$,
     consume that character and indicate success.
     Otherwise indicate failure.\dnsp \\    
\hline
$\tx{\textasciicircum[}$s$\tx{]}$\upsp 
   & where $s$ is a nonempty character string.
     If the text starts with a character \emph{not} appearing in $s$,
     consume that character and indicate success.
     Otherwise indicate failure.\dnsp \\    
\hline
$\tx{[}c_1\tx{-}\,c_2\tx{]}$\upsp
   & where $c_1,c_2$ are two characters.
     If the text starts with a character from the range 
     $c_1$~through~$c_2$, consume that character and indicate success.
     Otherwise indicate failure.\dnsp \\
\hline
\tx{\_}\upsp
   & where \tx{\_\,} is the underscore character.
     If there is a character ahead, consume it and indicate success.
     Otherwise (that is, at the end of input) indicate failure.\dnsp\\
\hline
\end{tabular}

\medskip
These expressions are analogous to terminals of a classical context-free
grammar, and are referred to as "terminals".
The remaining kinds of parsing expressions
invoke other expressions to do their job:

\medskip
\begin{tabular}{|c|p{0.83\linewidth}|}
\hline
$e\tx{?}$\upsp
   & Invoke the expression $e$ and indicate success whether it succeeded or not.\dnsp\\
\hline
$e\tx{*}$\upsp
   & Invoke the expression $e$ repeatedly as long as it succeeds.\newline
     Indicate success even if it did not succeed a single time.\dnsp\\
\hline
$e\tx{+}$\upsp
   & Invoke the expression $e$ repeatedly as long as it succeeds.\newline
     Indicate success if $e$ succeeded at least once.
     Otherwise indicate failure.\dnsp\\
\hline
$\tx{\&}e$\upsp
   & Invoke the expression $e$ and then
     reset the input as it was before the invocation of $e$\newline
     (do not consume anything).
     Indicate success if $e$ succeeded or failure if $e$ failed.\dnsp\\
\hline
$\tx{!}e$\upsp
   & Invoke the expression $e$ and then
     reset the input as it was before the invocation of $e$\newline
     (do not consume anything).
     Indicate success if $e$ failed or failure if $e$ succeeded.\dnsp\\     
\hline
$e_1 \ldots e_n$\upsp
   & Invoke expressions $e_1,\ldots,e_n$, in this order,
     as long as they succeed.\newline
     Indicate success if all succeeded. 
     Otherwise reset input as it was before the invocation of $e_1$\newline
     (do not consume anything) and indicate failure.\dnsp\\
\hline
$e_1\,\tx{/}\ldots\,\tx{/}\,e_n$\upsp
   & Invoke expressions $e_1,\ldots,e_n$, in this order,
     until one of them succeeds.\newline 
     Indicate success if one of expressions succeeded.
     Otherwise indicate failure.\dnsp\\
\hline
\end{tabular}

\medskip
The expressions $e, e_1, \ldots, e_n$ above can be specified either explicitly
or by name (the way of naming expressions will be explained in a short while).
An expression specified explicitly within another expression
with the same or higher precedence (see below)
must be enclosed in parentheses.
The following table summarizes all forms of parsing expressions.

% ---------------------------------------------------------------------
\begin{center}
\begin{tabular}{|c|l|c|} \hline
expression & name & precedence\upsp\dnsp \\ \hline
$\tx{"}s\tx{"}$  & String Literal \upsp & 5 \\
$\tx{[}s\tx{]}$  & Character Class & 5 \\
$\tx{\textasciicircum[}s\tx{]}$  & Not Character Class & 5 \\
$\tx{[}c_1\tx{-}\,c_2\tx{]}$ & Character Range & 5 \\
\textbf{\_} & Any Character & 5 \\
$e\tx{?}$  & Optional & 4 \\
$e\tx{*}$  & Iterate & 4 \\
$e\tx{+}$  & One or More & 4 \\
$\tx{\&}e$ & Ahead & 3 \\
$\tx{!}e$  & Not Ahead & 3 \\
$e_1 \ldots e_n$  & Sequence & 2 \\
$e_1\,\tx{/}\ldots\,\tx{/}\,e_n$  & Choice\dnsp & 1 \\ \hline
\end{tabular}
\end{center}
% ---------------------------------------------------------------------

\medskip
Backtracking takes place in the Sequence expression.
If $e_1,\ldots,e_i$ in $e_1\ldots e_i\ldots e_n$ succeed and consumed some input,
but then $e_{i\,+1}$ fails, the input is reset as it was before trying $e_1$;
the whole expression fails.
If this expression was invoked from a Choice expression (and was not the last there),
Choice has an opportunity to try another alternative on the same input.

However, the opportunities to try another alternative are limited:
once $e_i$ in the Choice \tx{$e_1/\ldots/e_i/\ldots /e_n$} succeeded, 
none of the alternatives $e_{i\,+1},\ldots e_n$ will ever be tried on the same input,
even if the parse fails later on.
This is the limited backtracking.

%----------------------------------------------------------------------
\subsection{The grammar\label{TheGram}}
%----------------------------------------------------------------------

Parsing Expression Grammar is a list of one or more "rules" of the form:
%
\begin{equation*}
\textit{name}\quad \tx{=}\quad \textit{expr} \;\;\tx{;}
\end{equation*}
%
where \textit{expr} is a parsing expression,
and \textit{name} is a name given to it.
The \textit{name} is a string of one or more letters (\tx{a-z}, \tx{A-Z}) and/or digits,
starting with a letter. 
White space is allowed everywhere except inside names.
Comments starting with a double slash and extending to the end of a line are also allowed.

The order of the rules does not matter, except that the expression specified first
is the "start expression", invoked at the start of the parser.

\medskip
A specific grammar may look like this:

\smallskip
\small
\begin{Verbatim}[frame=single,framesep=2mm,samepage=true,xleftmargin=15mm,xrightmargin=15mm,baselinestretch=0.8]
   Sum    = Number ("+" Number)* !_ ;
   Number = [0-9]+ ;
\end{Verbatim}
\normalsize
%
It consists of two named expressions: \tx{Sum} and \tx{Number}.
They define a parser consisting of
two procedures named \tx{Sum} and \tx{Number}.
The parser starts by invoking \tx{Sum}.
The \tx{Sum} invokes \tx{Number}, and if this succeeds,
repeatedly invokes \tx{("+" Number)} as long as it succeeds.
Finally, \tx{Sum}
invokes a sub-procedure for the predicate "\verb#!_#",
which succeeds only if it does not see any character ahead --
that is, only at the end of input.
The \tx{Number} reads digits in the range from 
0 through 9 as long as it succeeds,
and is expected to find at least one such digit.
%
One can easily see that the parser
accepts strings like "\tx{2+2}",  "\tx{17+4711}", or "\tx{2}";
in general, one or more integers separated by "\tx{+}".

It is quite obvious that
the expression $e$ in $e\tx{*}$ and $e\tx{+}$ must never succeed
in consuming an empty string;
otherwise, the parser could run into an infinite loop.
\Mouse\ checks this and gives you a warning if needed.

          % 2 Parsing Expression Grammar

\input{GetStarted}   % 3 Getting started

\input{FirstSteps}   % 4 The first steps

\input{Semantics}    % 5 Adding semantics

\input{RHS}          % 6 Understanding the "right-hand side"

\input{Realistic}    % 7 Getting more realistic

\input{Floating}     % 8 Let's go floating

\input{Backtracking} % 9 What about backtracking?

\input{NotPackRat}   % 10 A mouse, not a pack rat

\input{FullArith}    % 11 Full arithmetic

\input{Tree}         % 12 Want a tree?

\input{Calculator}   % 13 Calculator with memory

\input{Errors}       % 14 Get error handling right

\input{FileInput}    % 15 Input from file

\input{Recovery}     % 16 Error recovery

%HHHHHHHHHHHHHHHHHHHHHHHHHHHHHHHHHHHHHHHHHHHHHHHHHHHHHHHHHHHHHHHHHHHHHHHHHH

\section{Left Recursion}\label{LeftRec}

%HHHHHHHHHHHHHHHHHHHHHHHHHHHHHHHHHHHHHHHHHHHHHHHHHHHHHHHHHHHHHHHHHHHHHHHHHH

Looking at specifications of programming langauges
you probably noticed that syntax of artithmetic expressions 
is often defined in a different way.
A sum of integers, such as in Section~\ref{TheGram},
may be defined like this:
\smallskip
\small
\begin{Verbatim}[frame=single,framesep=2mm,samepage=true,xleftmargin=15mm,xrightmargin=15mm,baselinestretch=0.8]
   Sum    = Sum "+" Number / Number ;
   Number = [0-9]+ ;
\end{Verbatim}
\normalsize
But this does not work as PEG. 
Procedure \Sum\ would invoke itself repeatedly on the same input, in an infinite descent.
We say that \Sum\ is \emph{left-recursive}.

The above, on the other hand, makes perfect sense when read as EBNF,
and defines \Sum\ as one or more \Number\,s separated by \tx{"+"}.

Beginning with Version 2.0, \Mouse\ accepts such "pseudo-PEG" expressions
and treats them as EBNF syntax definitions.
The way of defining semantics is unchanged.
The left-recursive version of example from Section~\ref{Semantics} may appear like this:

\smallskip 
\small
\begin{Verbatim}[frame=single,framesep=2mm,samepage=true,xleftmargin=15mm,xrightmargin=15mm,baselinestretch=0.8]
   Compute = Sum !_ {print} ;
   Sum     = Sum "+" Number {sum}
           / Number {pass} ;
   Number  = [0-9]+ {number} ;
\end{Verbatim}
\normalsize

It was necessary to add \Compute\ because you do not want to print the result
after each occurrence of \Sum.
The semantic action \Numbera\ is the same as in Section~\ref{Semantics}.
Action \tx{pass()} just passes the value from \Number\ to \Sum:

\smallskip 
\small
\begin{Verbatim}[frame=single,framesep=2mm,samepage=true,xleftmargin=15mm,xrightmargin=15mm,baselinestretch=0.8]
  //-------------------------------------------------------------------
  //  Sum = Number
  //          0
  //-------------------------------------------------------------------
  void pass()
    { lhs().put(rhs(0).get()); }
\end{Verbatim}
\normalsize

\newpage
Action \Suma\ does the actual computation
and \Printa\ prints the result.

\smallskip 
\small
\begin{Verbatim}[frame=single,framesep=2mm,samepage=true,xleftmargin=15mm,xrightmargin=15mm,baselinestretch=0.8]
  //-------------------------------------------------------------------
  //  Sum = Sum "+" Number
  //         0   1    2
  //-------------------------------------------------------------------
  void sum()
    { lhs().put((Integer)rhs(0).get() + (Integer)rhs(2).get()); }
\end{Verbatim}
\normalsize

You find the grammar and semantics file in \tx{example2R};
copy it to the \tx{work} directory.
Generating the parser may look like this:

\small
\begin{Verbatim}[samepage=true,xleftmargin=15mm,baselinestretch=0.8]
java mouse.Generate -P myParser -G myGrammar.txt -S mySemantics
Parsing procedures:
2 rules
0 inner
3 terminals
4 procedures for 1 left-recursion class(es)
\end{Verbatim}
\normalsize

You may then try the generated parser:
  
\small
\begin{Verbatim}[samepage=true,xleftmargin=15mm,baselinestretch=0.8]
java mouse.TryParser -P myParser
> 17+4711
4728
> 12+2+3
17
>
\end{Verbatim}
\normalsize

\Mouse\ does this trick by generating behind the scenes
a Parsing Expression Grammar that is equivalent to your PEG
and invokes the semantic actions in the correct context.
Looking at the generated parser, you can see that left-recursive expressions
have special procedures whose names begin with "\tx{\$}".

The generated grammar is not left-recursive 
under condition that your PEG does not contain "cycles".
A cycle means that procedure can call itself without consuming any input.
Here is an example: 

\smallskip 
\small
\begin{Verbatim}[frame=single,framesep=2mm,samepage=true,xleftmargin=15mm,xrightmargin=15mm,baselinestretch=0.8]
   A = B "a"? / "a" ;
   B = A "b"? / "b" ;
\end{Verbatim}
\normalsize

as \tx{"a"?} and \tx{"b"?} can succeed without consuming anything,
\tx{A} can call \tx{B}
and then call \tx{A} with nothing being consumed.

The support for left recursion is restricted to Choice and Sequence expressions.
In addition, the first expression in recursive Sequence must never accept an empty string.
The semantic action "on failure" (described later) and Boolean action (see \ref{BoolAct})
are not suported for left-recursive rules.

In directory \tx{example4R} you find the grammar of Example~4 from Section~\ref{Floating}
rewritten in left-recursive way:
\smallskip 
\small
\begin{Verbatim}[frame=single,framesep=2mm,samepage=true,xleftmargin=15mm,xrightmargin=15mm,baselinestretch=0.8]
   Compute = Space Sum !_ {print} ; 
   Sum     = Sum AddOp Number {sum}
           / Number {pass} ;
   Number  = Sign Digits? "." Digits Space {fraction}
           / Sign Digits Space {integer} ; 
   Sign    = ("-" Space)? ;
   AddOp   = [-+] Space ;
   Digits  = [0-9]+ ;
   Space   = " "* ;
\end{Verbatim}
\normalsize

As in the preceding case, it was necessary to add \Compute.
\Sign\ had to be moved to \Number\
so that it applies only to the first term of the sum.
You find semantic actions in the same directory.

Directory \tx{example8R} contains left-recursive version of Example~8 from Section~\ref{errors}. 

Directory \tx{example11} contains examples of left-recursive grammars found in the literature.




      % 17 Left recursion

\newpage
%HHHHHHHHHHHHHHHHHHHHHHHHHHHHHHHHHHHHHHHHHHHHHHHHHHHHHHHHHHHHHHHHHHHHHHHHHH

\section{Extensions to PEG\label{Extensions}}

%HHHHHHHHHHHHHHHHHHHHHHHHHHHHHHHHHHHHHHHHHHHHHHHHHHHHHHHHHHHHHHHHHHHHHHHHHH

\Mouse\ accepts the following four forms of expression 
that are not part of the standard PEG:

\medskip
\begin{tabular}{|c|p{0.85\linewidth}|}
\hline
\ \ $e_1\tx{*+}\,e_2$\ \ \upsp
   & Shorthand for \tx{(!}$e_2\,e_1$\tx{)*}\,$e_2$:
     iterate $e_1$ until $e_2$.\dnsp\\
\hline
$e_1\tx{++}\,e_2$\upsp\newline
   & Shorthand for \tx{(!}$e_2\,e_1$\tx{)+}\,$e_2$:
     iterate $e_1$ at least once until $e_2$.\dnsp\\
\hline
\ \ $e_1\tx{:}\,e_2$\ \ \upsp
   & Invoke $e_1$.
     If it succeeds, invoke $e_2$ on the string consumed by~$e_1$.
     Indicate succes if it consumes exactly that string.
     Otherwise reset input as it was before the invocation of $e_1$
     (do not consume anything) and indicate failure.\dnsp\\
\hline
\ \ $e_1\tx{:!}\,e_2$\ \ \upsp
   & Invoke $e_1$.
     If it succeeds, invoke $e_2$ on the string consumed by~$e_1$.
     Indicate succes if does not consume exactly that string.
     Otherwise reset input as it was before the invocation of $e_1$
     (do not consume anything) and indicate failure.\dnsp\\
\hline
\end{tabular}

\bigskip
Using "\tx{*+}" and "\tx{++}", the definitions

\small
\begin{Verbatim}[samepage=true,xleftmargin=15mm,baselinestretch=0.8]
Comment = "/*" (!"*/" _)* "*/" ;
Input   = (!EOF Line)+ EOF ;
String  = ["] (!["] Char)+ ["] ;
\end{Verbatim}
\normalsize

are rewritten as 

\small
\begin{Verbatim}[samepage=true,xleftmargin=15mm,baselinestretch=0.8]
Comment = "/*" _*+ "*/" ;
Input   = Line++ EOF ;
String  = ["] Char++ ["] ;
\end{Verbatim}
\normalsize

which is not only easier to read, but also has a better implementation.

\medskip
The operators "\tx{:}" and "\tx{:!}" are intended to simplify definition of reserved words.

Suppose you want to define \tx{Identifier} as any sequence 
of letters other than \tx{"int"} or \tx{"float"}.
Using pure PEG, you can do it as follows, 
with \tx{INT} and \tx{FLOAT} used to denote these keywords in your syntax:

\small
\begin{Verbatim}[samepage=true,xleftmargin=15mm,xrightmargin=15mm,baselinestretch=0.8]
Identifier = !Keyword Letter+ ;
Keyword = ("int"/"float") !Letter ;
INT   = "int" !Letter ;
FLOAT = "float" !Letter ;
\end{Verbatim}
\normalsize

The predicate \tx{!Letter} is needed because
\tx{!("int"/"float")} fails on strings like \tx{internal} or \tx{floater},
which are thus not recognized as identifiers,
while their prefixes are falsely recognized as keywords.

Using the "colon" operators you define the same like this:

\small
\begin{Verbatim}[samepage=true,xleftmargin=15mm,xrightmargin=15mm,baselinestretch=0.8]
Word = Letter+ ;
Identifier = Word:!("int"/"float") ;
INT   = Word:"int" ;
FLOAT = Word:"float" ;
\end{Verbatim}
\normalsize

It is more clear and, more important, facilitates the inspection by \textsl{PEG Explorer}.

\medskip
All four operators have precedence 4 (cf. table in Section \ref{PEG}).

To obtain the "right-hand side" described in Section \ref{RHS},
replace \tx{*+} and \tx{++} by, respectively, \tx{*} and \tx{+} 
in the construction outlined there.

The second expression of  $e_1$\tx{:}$e_2$ and $e_1\tx{:!}e_2$ does not appear in the "right-hand side".
The expression together with the preceding operator should be removed in the desctibed construction.

Note: reserved words can also be defined with the help of Boolean actions, described under \ref{BoolAct}, 
but their definition is not visible in the syntax and is beyond the reach of \textsl{PEG Explorer}.
This is, however, the only way if the keywords depend on the preceding context like in,
for example, languages C and C++.

   % 18 Extensions to PEG

%HHHHHHHHHHHHHHHHHHHHHHHHHHHHHHHHHHHHHHHHHHHHHHHHHHHHHHHHHHHHHHHHHHHHHHHHHH

\section{Checking the grammar\label{Checking}}

%HHHHHHHHHHHHHHHHHHHHHHHHHHHHHHHHHHHHHHHHHHHHHHHHHHHHHHHHHHHHHHHHHHHHHHHHHH

Before generating the parser, \Mouse\ performs a number of checks on the grammar,
and issues messages about its findings.
Using the \Mouse\ tool \tx{TestPEG},
you can perform the same checking and obtain the same diagnostics
without generating the parser.
By specifying option \tx{-D} to the tool, you can also display the grammar.

As an example, below is the result of
\tx{TestPEG} for the grammar:
\small
\begin{Verbatim}[frame=single,framesep=2mm,samepage=true,xleftmargin=15mm,xrightmargin=15mm,baselinestretch=0.8]
  A = (!"a")* !_ / A "b"? ;
  B = B "a" ;
\end{Verbatim}
\normalsize

\medskip
\small
\begin{Verbatim}[samepage=true,xleftmargin=15mm,baselinestretch=0.8]
C:\Users\Giraf\Work>java mouse.TestPEG -G bad.peg -DL
Error: B is void.
Error: B "a" is void.
Error: the grammar has cycle inolving A.
Info: recursion class of B is not used.
Error: argument of (!"a")* is nullable.

  2 rules
  6 unnamed
  4 terminals
  4 left-recursive expressions in 2 classes

Rules
  A = (!"a")* !_ / A "b"? ;   //  1 rec class A
  B = B "a" ;   //  v rec class B

Inner
  (!"a")* !_   //  1ef
  (!"a")*   //  0
  !"a"   //  0f
  A "b"?   //  1 rec class A
  "b"?   //  01
  B "a"   //  v rec class B

Terminals
  "a"   //  1f
  !_   //  1ef
  "b"   //  1f
  "a"   //  1f
\end{Verbatim}
\normalsize

An expression being \emph{void} means that it does not generate any string, in the EBNF sense.
\newline
Thus,  \tx{B = B "a"} is void because it can only produce an infinite chain of replacements.
\newline
The notion  of \emph{cycle} was explained in Section~\ref{LeftRec}.
\newline
The symbols appearing as comments on the right have these meanings:
\ul
\item[] \tx{0} = may consume null string;
\item[] \tx{1} = may consume non-null string;
\item[] \tx{f} = may fail;
\item[] \tx{e} = ends processing;
\item[] \tx{v} = void.
\eul


Comment "\tx{rec class A}"  means that expression belongs to the \emph{recursion class} containing \tx{A}.
Recursion classes are sets of left-recursive expressions 
that interact only among themselves.
\tx{TestPEG} with option \tx{-L} displays information about recursion classes:

\small
\begin{Verbatim}[samepage=true,xleftmargin=15mm,baselinestretch=0.8]
java mouse.TestPEG -G bad.peg -L
Recursion class A
  members:
    A (entry) (exit)
    A "b"?
  seeds:
    (!"a")* !_ (in A)

Recursion class B
  members:
    B
    B "a"
  seeds:
\end{Verbatim}
\normalsize

An \emph{exit} of recursion class is 
a Choice expression with at least one alternative outside the class.
This alternative is a \emph{seed} that terminates the recursion.
An \emph{entry} to the class is the expression that starts left recursion.

\medskip
You may also want to find out how your grammar is affected by limited backtracking.
\newline
An interactive tool, the \textsl{PEG Explorer} assists you in checking the condition (*)
stated in Section~\ref{back}.
It is described in \url{http://mousepeg.sourceforge.net/explorer.htm}\,,
and is invoked as \tx{mouse.ExplorePEG}.




     % 19 Checking the grammar

\input{Features}     % 20 Miscellaneous features

\input{Deploying}    % 21 Deploying

\appendix

\newpage
%HHHHHHHHHHHHHHHHHHHHHHHHHHHHHHHHHHHHHHHHHHHHHHHHHHHHHHHHHHHHHHHHHHHHHHHHHH

\section{Appendix: The grammar of \Mouse\ PEG}

%HHHHHHHHHHHHHHHHHHHHHHHHHHHHHHHHHHHHHHHHHHHHHHHHHHHHHHHHHHHHHHHHHHHHHHHHHH

\small 
\begin{Verbatim}[frame=single,framesep=2mm,samepage=true,xleftmargin=10mm,xrightmargin=10mm,baselinestretch=0.8]

  Grammar   = Space (Rule/Skip)*+ EOT {Grammar} ;
  Rule      = Name EQUAL RuleRhs DiagName? SEMI {Rule} ~{Error} ;
  Skip      = SEMI             
            / _++ (SEMI/EOT) ; 
  RuleRhs   = Sequence Actions (SLASH Sequence Actions)* {} <right-hand side> ;
  Choice    = Sequence (SLASH Sequence)* {} ;
  Sequence  = Prefixed+ {} ;
  Prefixed  = (AND / NOT) Suffixed {Prefix}
            / Suffixed {Pass} ;
  Suffixed  = Primary (STARPLUS / PLUSPLUS / IS / ISNOT) Primary {Infix}
            / Primary (QUERY / STAR / PLUS) {Suffix}
            / Primary {Pass} ;
  Primary   = Name      {Resolve}
            / LPAREN Choice RPAREN {Pass2}
            / ANY       {Any}
            / StringLit {Pass}
            / Range     {Pass}
            / CharClass {Pass} ;
  Actions   = OnSucc OnFail {} ;
  OnSucc    = (LWING AND? Name? RWING)? {} ;
  OnFail    = (TILDA LWING Name? RWING)? {} ;
  Name      = Letter (Letter / Digit)* Space {} ;
  DiagName  = "<" Char++ ">" Space {} ;
  StringLit = ["] Char++ ["] Space {} ;
  CharClass = ("[" / "^[") Char++ "]" Space {} ;
  Range     = "[" Char "-" Char "]" Space {} ;
  Char      = Escape    {Pass}
            / ^[\r\n\\] {Char} ;
  Escape    = "\\u" HexDigit HexDigit HexDigit HexDigit {Unicode}
            / "\\t" {Tab}
            / "\\n" {Newline}
            / "\\r" {CarRet}
            / !"\\u""\\"_ {Escape} ;
  Letter    = [a-z] / [A-Z] ;
  Digit     = [0-9] ;
  HexDigit  = [0-9] / [a-f] / [A-F] ;
  AND       = [&]     Space  <&> ;
  NOT       = [!]     Space  <!> ;
  QUERY     = [?]     Space  <?> ;
  STAR      = [*]![+] Space  <*> ;
  PLUS      = [+]![+] Space  <+> ;
  STARPLUS  = "*+"    Space <*+> ;
  PLUSPLUS  = "++"    Space <++> ;
  IS        = [:]![!] Space  <:> ;
  ISNOT     = ":!"    Space <:!> ;
  EQUAL     = [=]     Space  <=> ;
  LPAREN    = [(]     Space  <(> ;
  RPAREN    = [)]     Space  <)> ;
  LWING     = [{]     Space  <{> ;
  RWING     = [}]     Space  <}> ;
  SEMI      = [;]     Space  <;> ;
  SLASH     = [/]     Space  </> ;
  TILDA     = [~]     Space  <~> ;
  ANY       = [_]     Space  <_> ;
  Space     = ([ \r\n\t] / Comment)*  ;
  Comment   = "//" _*+ EOL ;
  EOL       = [\r]? [\n] / !_  <end of line> ;
  EOT       = !_               <end of text> ;

\end{Verbatim}
\normalsize
    % The grammar of Mouse PEG

\input{AppendixB}    % Helper methods

\input{AppendixC}    % Your parser class

\input{AppendixD}    % Mouse tools

\newpage
    
%HHHHHHHHHHHHHHHHHHHHHHHHHHHHHHHHHHHHHHHHHHHHHHHHHHHHHHHHHHHHHHHHHHHHHHHHHH
    
% Bibliography

%HHHHHHHHHHHHHHHHHHHHHHHHHHHHHHHHHHHHHHHHHHHHHHHHHHHHHHHHHHHHHHHHHHHHHHHHHH

\bibliographystyle{acm}
\bibliography{Bib}
\end{document}
